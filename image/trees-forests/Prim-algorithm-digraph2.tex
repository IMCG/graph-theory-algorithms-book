%%%%%%%%%%%%%%%%%%%%%%%%%%%%%%%%%%%%%%%%%%%%%%%%%%%%%%%%%%%%%%%%%%%%%%%%%%%
%% This file is part of the book
%%
%% Algorithmic Graph Theory
%% http://code.google.com/p/graph-theory-algorithms-book/
%%
%% Copyright (C) 2009, 2010 Minh Van Nguyen <nguyenminh2@gmail.com>
%%
%% See the file COPYING for copying conditions.
%%%%%%%%%%%%%%%%%%%%%%%%%%%%%%%%%%%%%%%%%%%%%%%%%%%%%%%%%%%%%%%%%%%%%%%%%%%

\subfigure[Original undirected graph.]{
\begin{tikzpicture}
[lineDecorate/.style={-,thick},%
  nodeDecorate/.style={shape=circle,inner sep=2pt,draw,thick}]
%% nodes or vertices
\foreach \nodename/\x/\y in {
  0/0/0, 1/3/1.5, 2/6/3, 3/0/3, 4/3/4.5, 5/0/6, 6/3/7.5}
{
  \node (\nodename) at (\x,\y) [nodeDecorate] {$\nodename$};
}
%% edges or lines
\tikzstyle{EdgeStyle}=[-,thick]
\tikzstyle{LabelStyle}=[fill=white]
\foreach \startnode/\endnode/\weight in {
  0/1/7, 0/3/5, 1/2/8, 1/3/9, 1/4/7, 2/4/5, 3/4/15, 3/5/6, 4/5/8,
  4/6/9, 5/6/11}
{
  \Edge[label=$\weight$](\startnode)(\endnode)
}
\end{tikzpicture}
}
\qquad
%%
%%
\subfigure[Minimum spanning tree.]{
\begin{tikzpicture}
[lineDecorate/.style={-,thick},%
  nodeDecorate/.style={shape=circle,inner sep=2pt,draw,thick}]
%% nodes or vertices
\foreach \nodename/\x/\y in {
  0/0/0, 1/3/1.5, 2/6/3, 3/0/3, 4/3/4.5, 5/0/6, 6/3/7.5}
{
  \node (\nodename) at (\x,\y) [nodeDecorate] {$\nodename$};
}
%% edges or lines
\tikzstyle{EdgeStyle}=[-,thick]
\tikzstyle{LabelStyle}=[fill=white]
\foreach \startnode/\endnode/\weight in {
  0/1/7, 1/4/7, 3/0/5, 4/2/5, 4/6/9, 5/3/6}
{
  \Edge[label=$\weight$](\startnode)(\endnode)
}
\end{tikzpicture}
}
