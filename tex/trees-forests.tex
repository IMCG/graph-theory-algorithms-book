%%-----------------------------------------------------------------------%%
%%--- Trees and Forests -------------------------------------------------%%

\chapter{Trees and Forests}


%%-----------------------------------------------------------------------%%
%%--- Properties of trees -----------------------------------------------%%

\section{Properties of trees}

\begin{itemize}
\item trees and acyclic graphs; leaves

\item forests
\end{itemize}

\begin{theorem}
If $T = (V, E)$ is a graph with $n$ vertices, then the following
statements are equivalent:
\begin{enumerate}
\item $T$ is a tree.

\item $T$ contains no cycles and has $n - 1$ edges.

\item $T$ is connected and has $n - 1$ edges.

\item Every edge of $T$ is a cut-edge.

\item For any $u,v \in V$, there is exactly one $u$-$v$ path.

\item For any new edge $e$, the join $T + e$ has exactly one cycle.
\end{enumerate}
\end{theorem}


%%-----------------------------------------------------------------------%%
%%--- Minimum spanning trees --------------------------------------------%%

\section{Minimum spanning trees}

\begin{itemize}
\item spanning trees

\item minimum-cost spanning trees

\item Kruskal's algorithm

\item Prim's algorithm

\item Bor\r{u}vka's algorithm
\end{itemize}


%%-----------------------------------------------------------------------%%
%%--- Binary trees ------------------------------------------------------%%

\section{Binary trees}

\begin{itemize}
\item binary codes

\item Huffman codes

\item Huffman algorithm
\end{itemize}


%%-----------------------------------------------------------------------%%
%%--- Tree traversals ---------------------------------------------------%%

\section{Tree traversals}

\begin{itemize}
\item stacks and queues

\item level-order traversal

\item pre-order traversal

\item post-order traversal

\item in-order traversal
\end{itemize}


%%-----------------------------------------------------------------------%%
%%--- Binary search trees -----------------------------------------------%%

\section{Binary search trees}

\begin{itemize}
\item records and keys

\item searching a binary search tree (BST)

\item inserting into a BST

\item deleting from a BST

\item traversing a BST

\item sorting using BST
\end{itemize}
