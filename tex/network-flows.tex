%%%%%%%%%%%%%%%%%%%%%%%%%%%%%%%%%%%%%%%%%%%%%%%%%%%%%%%%%%%%%%%%%%%%%%%%%%%
%% This file is part of the book
%%
%% Algorithmic Graph Theory
%% http://code.google.com/p/graph-theory-algorithms-book/
%%
%% Copyright (C) 2009--2011 Minh Van Nguyen <nguyenminh2@gmail.com>
%%
%% See the file COPYING for copying conditions.
%%%%%%%%%%%%%%%%%%%%%%%%%%%%%%%%%%%%%%%%%%%%%%%%%%%%%%%%%%%%%%%%%%%%%%%%%%%

\chapter{Network Flows}
\label{chap:network_flows}

See Jungnickel~\cite{Jungnickel2008}, and chapter~12 of Gross and
Yellen~\cite{GrossYellen1999}.


%%%%%%%%%%%%%%%%%%%%%%%%%%%%%%%%%%%%%%%%%%%%%%%%%%%%%%%%%%%%%%%%%%%%%%%%%%%

\section{Flows and cuts}

\begin{itemize}
\item single source-single sink networks

\item feasible networks

\item maximum flow and minimum cut
\end{itemize}


Let $G=(V,E,i,h)$ be an unweighted multidigraph, as in
Definition \ref{def:introduction:graph-version3}.

If $F$ is a field such as ${\mathbb{R}}$ or $GF(q)$ or a 
ring such as ${\mathbb{Z}}$, let 

\[
C^0(G, F)=\{f:V\to F\},\ \ \ \ C^1(G, F)=\{f:E\to F\},
\]
be the sets of $F$-valued functions defined on $V$ and
$E$, respectively. 
\index{$C^0(G, F)$}
\index{$C^1(G, F)$}
If $F$ is a field then these are $F$-inner
product spaces with inner product

\begin{equation}
\label{eqn:inner-product}
(f,g)=\sum_{x\in X} f(x)g(x), \ \ \ \ \ (X=V,\ \ {\rm resp.}\ \ X=E),
\end{equation}
and

\[
\dim C^0(G,F)=|V|,\ \ \ \dim C^1(G,F)=|E|.
\]
If you index the sets $V$ and $E$ in some arbitrary but fixed way
and define, for $1\leq i\leq |V|$ and
$1\leq j\leq |E|$,

\[
f_i(v)=
\left\{
\begin{array}{ll}
1,& v = v_i,\\
0, & {\rm otherwise},
\end{array}
\right.
\ \ \ \ \ 
g_j(e)=
\left\{
\begin{array}{ll}
1,& e = e_j,\\
0, & {\rm otherwise},
\end{array}
\right.
\]
then ${\mathcal F}=\{f_i\}\subset C^0(G,F)$ is a basis
of $ C^0(G,F)$
and ${\mathcal G}=\{g_j\}\subset C^1(G,F)$ is a basis of $C^1(G,F)$.

We order the edges

\[
E = \{e_1,e_2,\dots, e_{|E|}\},
\]
in some arbitrary but fixed way.
A {\it vector representation} (or {\it characteristic vector} or
{\it incidence vector}) of a subgraph $G'=(V,E')$ of $G=(V,E)$,
$E'\subset E$, is a binary ${|E|}$-tuple

\[
{\rm vec}(G')=(a_1,a_2,\dots, a_{|E|}) \in GF(2)^{|E|},
\]
where 

\[
a_i=a_i(E')=
\left\{
\begin{array}{ll}
1,&{\rm if \ }e_i\in E',\\
0,&{\rm if \ }e_i\notin E'.
\end{array}
\right.
\]
In particular, this defines a mapping

\[
{\rm vec}:\{ {\rm subgraphs\ of\ }G=(V,E)\}\to GF(2)^{|E|}.
\]


For any non-trivial partition 

\[
V = V_1\cup V_2,\ \ \ \ \ V_i\not= \emptyset, \ \ \ V_1\cap V_2= \emptyset, 
\]
the set of all edges $e=(v_1,v_2)\in E$, with $v_i\in V_i$ ($i=1,2$),
is called a {\it cocycle}\footnote{Also called an {\it edge cut subgraph} 
or {\it disconnecting set} or
{\it seg} or {\it edge cutset}.} of $G$. 
A cocycle with a minimal set of edges is a  
{\it bond} (or {\it cut set}) of $G$.  
An {\it Euler subgraph} is either a cycle
or a union of edge-disjoint cycles.
\index{bond}
\index{Euler subgraph}
\index{cocyle}
\index{edge cut subgraph}
\index{seg}
\index{cut set}
\index{edge cutset}

The set of cycles of $G$ is denoted $Z(G)$ and the set of
cocycles is denoted $Z^*(G)$.

The $F$-vector space spanned by the vector representations of 
all the cycles is called the {\it cycle space} of $G$, denoted
${\mathcal Z}(G)={\mathcal Z}(G,F)$. 
\index{cycle space}
This is the kernel of the incidence 
matrix of $G$ (\S 14.2 in Godsil and Royle \cite{GodsilRoyle2004}).
Define

\[
\begin{array}{c}
D:C^1(G,F)\to C^0(G,F),\\
(Df)(v) = \sum_{h(e)=v} f(e) - \sum_{t(e)=v} f(e).
\end{array}
\]
With respect to these bases ${\mathcal F}$ and
${\mathcal G}$, the matrix
representing the linear transformation $D:C^1(G,F)\to C^0(G,F)$
is the incidence matrix. An element of the kernel of $D$ is 
sometimes called a {\it flow} (see Biggs \cite{Biggs1997})
or {\it circulation} (see below).
\index{flow}
\index{circulation}
Therefore, this is sometimes also referred to as the 
{\it space of flows} or the {\it circulation space}.
\index{flow space}
\index{circulation space}

It may be regarded as a subspace of $C^1(G,F)$ of dimension $n(G)$.
When $F$ is a finite field, sometimes\footnote{Jungnickel and 
Vanstone in \cite{JungnickelVanstone1997} call 
this the {\it even graphical code} of $G$.}
the cycle space is called the {\it cycle code of $G$}.
\index{cycle code}


Let $F$ be a field such as ${\mathbb{R}}$ or $GF(q)$, for some prime 
power $q$. Let $G$ be a digraph.
Some define a {\it circulation} (or {\it flow}) on $G=(V,E)$ 
\index{circulation}
to be a function

\[
f:E\to F,
\]
satisfying\footnote{Note: In addition, some authors add the 
condition $f(e)\geq 0$ - see e.g., Chung \cite{Chung2005}.}

\begin{itemize}
\item
$\sum_{u\in V,\ (u,v)\in E} f(u,v) = \sum_{w\in V,\ (v,w)\in E} f(v, w)$.
\end{itemize}
(Note: this is simply the condition that $f$ belongs to the
kernel of $D$.)

Suppose $G$ has a subgraph $H$ and 
$f$ is a circulation of $G$ such that $f$ is a constant function
on $H$ and $0$ elsewhere. We call such a circulation a
{\it characteristic function} of $H$.
For example, if $G$ has a cycle $C$ and if $f$ is the
characteristic function on $C$, then $f$ is a circulation.

The {\it circulation space} ${\mathcal C}$ is the $F$-vector space of 
circulation functions. 
The cycle space ``clearly'' may be identified with 
a subspace of the circulation space,
since the $F$-vector space spanned by the characteristic
functions of cycles may be identified with the cycle space of $G$. 
In fact, these spaces are isomorphic.
Under the inner product (\ref{eqn:inner-product}), i.e., 
\index{circulation space}

\begin{equation}
%\label{eqn:inner-product2}
(f,g)=\sum_{e\in E} f(e)g(e),
\end{equation}
this vector space is an inner product space.

\begin{example}
{\rm
This example is not needed but is presented for its independent
interest. Assume $G=(V,E)$ is a strongly connected directed graph.
Define the {\it transition probability matrix} $P$ for a digraph
\index{transition probability matrix}
$G$ by

\[
P(x,y)=\left\{
\begin{array}{ll}
d_x, & {\rm if}\ (x, y) \in E,\\
0, & {\rm otherwise},
\end{array}
\right.
\]
where $d_x$ denotes the out-degree.
The Perron-Frobenius Theorem states that there exists a unique 
left eigenvector $\phi$ such that (when regarded as a function
$\phi:V\to {\mathbb{R}}$) $\phi (v) > 0$, for all $v\in V$ and 
$\phi P  = \rho \phi$, where $\rho$ is the spectral radius of
$G$. We scale $\phi$ so that $\sum_{v\in V}  \phi(v) = 1$.
(This vector is sometimes called the {\it Perron vector}.)
 Let $F_\phi (u, v) = \phi(v)P (u, v)$. 
{\it Fact:} $F_\phi$ is a circulation.
For a proof, see F. Chung \cite{Chung2005}.
}
\end{example}

 If the edges of $E$ are indexed in some arbitrary but fixed way then
 a circulation function restricted to a subgraph $H$ of $G$
 may be identified with a vector representation of $H$,
 as described above. Thefore, the circulation functions gives a
 coordinate-free version of the cycle space.

The $F$-vector space spanned by the vector representations of 
all the segs is called the {\it cocycle space} (or the {\it cut space}) of $G$, 
denoted ${\mathcal Z}^*(G)={\mathcal Z}^*(G,F)$.
This is the column space of the transpose of the incidence 
matrix of $G$ (\S 14.1 in Godsil and Royle \cite{GodsilRoyle2004}). It may be 
regarded as a subspace of $C^1(G,F)$ of dimension the rank of $G$,
$r(G)$.
When $F$ is a finite field, sometimes the cocycle space is called 
the {\it cocycle code of $G$}.
\index{cocycle code}
\index{cocycle space}
\index{cut space}

\begin{lemma}
\label{lemma:cycle-cocycle}
{\rm
Under the inner product (\ref{eqn:inner-product})
on $C^1(G,F)$, the cycle space is orthogonal to the cocycle space.
}
\end{lemma}

\begin{proof}[Solution]
One proof follows from Theorem 8.3.1 in Godsil and Royle \cite{GodsilRoyle2004}.

Here is another proof.
By Theorem~2.3 in Bondy and Murty~\cite[p.27]{BondyMurty1976},
an edge of $G$ is an edge cut if and only if it
is contained in no cycle.
Therefore, the vector representation of any 
cocycle is supported on a set of indices which is disjoint from the 
support of the vector representation of any
cycle. Therefore, there is a basis of the cycle space which is 
orthogonal to a basis of the cocycle space.
\end{proof}

\begin{proposition}
{\rm
Let $F=GF(2)$.
The cycle code of a graph $G=(V,E)$ is a linear binary block code of 
length $|E|$,  dimension equal to the nullity of the graph, $n(G)$,
and minimum distance equal to the girth of $G$.
If $C\subset GF(2)^{|E|}$ is the cycle code associated to $G$  
and $C^*$ is the cocycle code associated to $G$ then
$C^*$ is the dual code of $C$. In particular,
the cocycle code of $G$ is a linear binary block code of length
$|E|$, and dimension $r(G)=|E|-n(G)$.
}
\end{proposition}

This follows from Hakimi-Bredeson \cite{HakimiBredeson1968}
(see also Jungnickel-Vanstone \cite{JungnickelVanstone1997})
in the binary case\footnote{It is likely true
in the non-binary case as well, but no
proof seems to be in the literature.}. 

\begin{proof}[Solution]
Let $d$ denote the minimum distance of the code $C$.
Let $\gamma$ denote the girth of $G$,
i.e., the smallest cardinality of a cycle in $G$.
If $K$ is a cycle in $G$ then the
vector ${\rm vec}(K)\in GF(2)^{|E|}$ is an element of the
cycle code $C\subset GF(2)^{|E|}$. This implies $d\leq \gamma$.

In the other direction, suppose $K_1$ and $K_2$ are
cycles in $G$ with associated support vectors 
$v_1={\rm vec}(K_1)$, $v_2={\rm vec}(K_2)$.
Assume that at least one of these cycles is a cycle
of minimum length, say $K_1$, so the weight of its corresponding 
support vector is equal to the girth $\gamma$.
The only way that 
${\rm wt}(v_1+v_2)<\min\{{\rm wt}(v_1),{\rm wt}(v_2)\}$ 
can occur is if $K_1$ and $K_2$ have some
edges in common. In this case, the vector
$v_1+v_2$ represents a subgraph 
which is either a cycle or it is a 
union of disjoint cycles. In either case, by
minimality of $K_1$, these new cycles must be at least as
long. Therefore, $d\geq \gamma$, as desired.

%Lemma \ref{lemma:cycle-cocycle} below.
\end{proof}

Consider a spanning tree $T$ of a graph $G$
and its complementary subgraph $\overline{T}$.
For each edge $e$ of $\overline{T}$ the graph $T\cup e$
contains a unique cycle. The cycles which arise in this way are called
the {\it fundamental cycles} of $G$, denoted $cyc(T,e)$.
\index{fundamental cycles}

%If $G$ is connected, one can construct the generator 
%matrix of the cycle code $C$ in standard form as follows. 
%Fix a spanning tree $T$ of $G$ and index the
%edges of $G$ so that the edges in the complementary subgraph $\overline{T}$ 
%(namely, $E(\overline{T}) = E(G)-E(T)$) occur before the edges of $T$. 
%If you define the generator matrix $A$ of $C$ to be that matrix whose
%rows are the vector representation of the fundamental cycles
%$cyc(T,e)$, as $e$ runs through the edges in $\overline{T}$
%in the order above, then you will find that $A$ is in standard form.



\begin{example}
\label{example:cycle-space}
{\rm
Consider the graph below, with edges labeled as indicated, together
with a spanning tree, depicted to its right, in 
Figure \ref{fig:cycle-space1}.


\begin{figure}[h!]
\begin{center}
\begin{tabular}{cc}
% this was done "manually"
\unitlength=0.620000pt
\begin{picture}(250.00,100.00)(-40.00,0.00)

\thinlines
\put(0.00,0.00){\line(1,0){50.00}} 
\put(17.00,30.00){$0$} 
\put(0.00,0.00){\line(1,1){50.00}} 
\put(25.00,-15.00){$1$} 
\put(50.00,0.00){\line(0,1){50.00}} 
\put(55.00,20.00){$2$} 
\put(50.00,0.00){\line(1,0){50.00}} 
\put(75.00,-15.00){$3$} 

\put(100.00,0.00){\line(1,0){50.00}} 
\put(117.00,30.00){$4$} 
\put(100.00,0.00){\line(1,1){50.00}}
\put(125.00,-15.00){$5$} 

\put(150.00,50.00){\line(1,0){50.00}} 
\put(175.00,57.00){$7$} 
\put(150.00,50.00){\line(0,-1){50.00}}
\put(155.00,20.00){$6$}  
\put(200.00,50.00){\line(0,-1){50.00}} 
\put(187.00,20.00){$8$}  
\put(225.00,57.00){$9$} 
\put(200.00,50.00){\line(1,0){50.00}} 

\put(232.00,17.00){$10$}  
\put(200.00,0.00){\line(1,1){50.00}}
\end{picture}

&
% this used jPicEdt
\unitlength=0.620000pt
\begin{picture}(550.00,100.00)(-80.0,0.00)

%\linethickness{0.9mm}
\put(0.00,0.00){\line(1,0){50.00}} 

%\linethickness{1.9mm}
\put(0.00,0.00){\line(1,1){50.00}} 
%\put(50.00,0.00){\line(0,1){50.00}} 

%\linethickness{0.9mm}
\put(50.00,0.00){\line(1,0){50.00}} 
\put(100.00,0.00){\line(1,0){50.00}} 

%\linethickness{1.9mm}
\put(100.00,0.00){\line(1,1){50.00}}

%\linethickness{0.9mm}
\put(150.00,50.00){\line(1,0){50.00}} 
%\put(150.00,50.00){\line(0,-1){50.00}} 
\put(200.00,50.00){\line(0,-1){50.00}} 
\put(200.00,50.00){\line(1,0){50.00}} 

\end{picture}
\end{tabular}
\caption{A graph and a spanning tree for it.}
\end{center}
\label{fig:cycle-space1}
\end{figure}

You can see from Figure \ref{fig:cycle-space1} that:

\begin{itemize}
\item
by adding edge $2$ to the tree, you get a cycle $0,1,2$ with vector
representation

\[
g_1 = (1,1,1,0,0,0,0,0,0,0,0),
\]
\item
by adding edge $6$ to the tree, you get a cycle $4,5,6$ with vector
representation

\[
g_2 = (0,0,0,0,1,1,1,0,0,0,0),
\]
\item
by adding edge $10$ to the tree, you get a cycle $8,9,10$ with vector
representation

\[
g_3 = (0,0,0,0,0,0,0,0,1,1,1).
\]
\end{itemize}
The vectors $\{g_1,g_2,g_3\}$ form a basis of the cycle space 
of $G$ over $GF(2)$.
}
\end{example}

The {\it cocycle space} of a graph $G$ (also known as 
the {\it bond space of $G$} or the {\it cut-set space} of $G$)
is the $F$-vector space spanned by the characteristic
functions of bonds. 


\begin{example}
\label{example:cocycle-space}
{\rm
Consider the graph below, with edges labeled as indicated, together
with an example of a bond, depicted to its right, in 
Figure \ref{fig:cocycle-space1}.


\begin{figure}[h!]
\begin{center}
\begin{tabular}{cc}
% this was done "manually"
\unitlength=0.620000pt
\begin{picture}(250.00,100.00)(-40.00,0.00)

\thinlines
\put(0.00,0.00){\line(1,0){50.00}} 
\put(17.00,30.00){$0$} 
\put(0.00,0.00){\line(1,1){50.00}} 
\put(25.00,-15.00){$1$} 
\put(50.00,0.00){\line(0,1){50.00}} 
\put(55.00,20.00){$2$} 
\put(50.00,0.00){\line(1,0){50.00}} 
\put(75.00,-15.00){$3$} 

\put(100.00,0.00){\line(1,0){50.00}} 
\put(117.00,30.00){$4$} 
\put(100.00,0.00){\line(1,1){50.00}}
\put(125.00,-15.00){$5$} 

\put(150.00,50.00){\line(1,0){50.00}} 
\put(175.00,57.00){$7$} 
\put(150.00,50.00){\line(0,-1){50.00}}
\put(155.00,20.00){$6$}  
\put(200.00,50.00){\line(0,-1){50.00}} 
\put(187.00,20.00){$8$}  
\put(225.00,57.00){$9$} 
\put(200.00,50.00){\line(1,0){50.00}} 

\put(232.00,17.00){$10$}  
\put(200.00,0.00){\line(1,1){50.00}}
\end{picture}

&
% this used jPicEdt
\unitlength=0.620000pt
\begin{picture}(550.00,100.00)(-80.0,0.00)

%\linethickness{0.9mm}
\put(0.00,0.00){\line(1,0){50.00}} 

%\linethickness{1.9mm}
\put(0.00,0.00){\line(1,1){50.00}} 
%\put(50.00,0.00){\line(0,1){50.00}} 

% moved bond segment to the end
\put(100.00,0.00){\line(1,0){50.00}} 

%\linethickness{1.9mm}
\put(100.00,0.00){\line(1,1){50.00}}

%\linethickness{0.9mm}
\put(150.00,50.00){\line(1,0){50.00}} 
%\put(150.00,50.00){\line(0,-1){50.00}} 
\put(200.00,50.00){\line(0,-1){50.00}} 
\put(200.00,50.00){\line(1,0){50.00}} 

\linethickness{0.9mm}
\put(50.00,0.00){\line(1,0){50.00}} 
\end{picture}
\end{tabular}
\caption{A graph and a bond of it.}
\end{center}
\label{fig:cocycle-space1}
\end{figure}

You can see from Figure \ref{fig:cocycle-space1} that:

\begin{itemize}
\item
by removing edge $3$ from the graph, you get a bond with vector
representation

\[
b_1 = (0,0,0,1,0,0,0,0,0,0,0),
\]
\item
by removing edge $7$ from the graph, you get a bond with vector
representation

\[
b_2 = (0,0,0,0,0,0,0,1,0,0,0),
\]
\item
by removing edges $0,1$ from the graph, you get a bond with vector
representation

\[
b_3 = (1,1,0,0,0,0,0,0,0,0,0),
\]
\item
by removing edges $1,2$ from the graph, you get a bond with vector
representation

\[
b_4 = (0,1,1,0,0,0,0,0,0,0,0),
\]
\item
by removing edges $4,5$ from the graph, you get a bond with vector
representation

\[
b_5 = (0,0,0,0,1,1,0,0,0,0,0),
\]
\item
by removing edges $4,6$ from the graph, you get a bond with vector
representation

\[
b_6 = (0,0,0,0,1,0,1,0,0,0,0),
\]
\item
by removing edges $8,9$ from the graph, you get a bond with vector
representation

\[
b_7 = (0,0,0,0,0,0,0,0,1,1,0),
\]
\item
by removing edges $9,10$ from the graph, you get a bond with vector
representation

\[
b_8 = (0,0,0,0,0,0,0,0,0,1,1).
\]
\end{itemize}
The vectors $\{b_1,b_2,b_3,b_4,b_5,b_6,b_7,b_8\}$ form a basis of the 
cocycle space of $G$ over $GF(2)$. 

Note that these vectors are orthogonal to the basis vectors of
the cycle space in Example \ref{example:cycle-space}. Note also that
the xor sum of two cuts is not a cut.For example, if you xor the bond
$4,5$ with the bond $4,6$ then you get the subgraph foormed by the 
edges $5,6$ and that is not a disconnecting cut of $G$.
}
\end{example}



%%%%%%%%%%%%%%%%%%%%%%%%%%%%%%%%%%%%%%%%%%%%%%%%%%%%%%%%%%%%%%%%%%%%%%%%%%%


\section{Chip firing games}

Chip firing games on graphs (which are just
pure fun) relate to ``abelian sandpile models'' from physics to 
``rotor-routing models'' from theoretical computer scientists
(designing efficient computer multiprocessor circuits) to ``self-organized
criticality'' (a subdiscipline of dynamical systems) to 
``algebraic potential theory'' on a graph \cite{Biggs1997} to cryptography
(via the Biggs cryptosystem). Moreover, it relates the concepts of the 
Laplacian of the graph to the tree number to the circulation space
of the graph to the incidence matrix, as well as many other
ideas. Some good references are \cite{Durgin2009}, \cite{PerkinsonEtAl2009},
\cite{Perlman2009}, \cite{HolroydEtAl2008} and \cite{Biggs1999}.
\index{chip-firing games}
\index{rotor-routing model}

\subsubsection{Basic set-up}

A {\it chip firing game} always starts with a directed
multigraph $G$ having no loops. A {\it configuration} is a 
vertex-weighting, i.e., a function
$s:V\to {\mathbb{R}}$. The players are represented by the
vertices of $G$ and the vertex-weights represent the number of
chips each player (represented by that vertex) has.
The initial vertex-weighting is called the {\it starting configuration} of $G$.
Let vertex $v$ have outgoing degree $d_+(v)$. If the weight of 
vertex $v$ is $\geq d_+(v)$ (so that player can afford to give away
all their chips) then that vertex is called {\it active}.
\index{active vertex}
\index{configuration}
\index{configuration!starting}
\index{chip firing game}

Here is some \sage/{\tt Python} code for determining the active vertices.

\begin{Verbatim}[fontsize=\scriptsize,fontfamily=courier,fontshape=tt,frame=single,label=\sage]

def active_vertices(G, s):
    """
    Returns the list of active vertices.

    INPUT:
     G - a graph
     s - a configuration (implemented as a list 
                          or a dictionary keyed on
                          the vertices of the graph)

    EXAMPLES:
        sage: A = matrix([[0,1,1,0,0],[1,0,1,0,0],[1,1,0,1,0],[0,0,0,0,1],[0,0,0,0,0]])
        sage: G = Graph(A, format = "adjacency_matrix", weighted = True)
        sage: s = {0: 3, 1: 1, 2: 0, 3: 1, 4: 1}
        sage: active_vertices(G, s)
        [0, 4]

    """
    V = G.vertices()
    degs = [G.degree(v) for v in V]
    active = [v for v in V if degs[V.index(v)]<=s[v]]
    return active

\end{Verbatim}
If $v$ is active then when you fire $v$ you must also change the
configuration. The new configuration $s'$ will satisfy
$s'(v) = s(v)-d_+(v)$ and $s'(v')=s(v')+1$ for each 
neighbor $v'$ of $v$. In other words, $v$ will give away one chip to
each of its $d_+(v)$ neighbors. If $x:V\to \{0,1\}^{|V|}\subset {\mathbb{R}}^{|V|}$ is
the representation vector (``characteristic function'') of a vertex
then this change can be expressed more compactly as
\index{chip-firing game}

\begin{equation}
\label{eqn:fire}
s'= s-L\circ x,
\end{equation}
where $L$ is the vertex Laplacian. It turns out that the column sums
of $L$ are all $0$, so this operation does not change the total
number of chips. We use the notation

\[ 
s\overset{v}{\to} s',
\] 
to indicate that the configuration $s'$ is the
result of firing vertex $v$ in configuration $s$.

\begin{example}
{\rm
Consider the graph

\vskip .2in
\begin{figure}[h!]

% this was done "manually"
\unitlength=0.620000pt
\begin{picture}(250.00,100.00)(-250.00,0.00)

\thinlines
\put(0.00,50.00){\line(1,0){100.00}} 
\put(-1.00,47.00){$\bullet$} 
\put(-13.00,50.00){$0$} 
\put(0.00,50.00){\line(1,1){50.00}} 
\put(50.00,112.00){$1$} 
\put(45.00,97.00){$\bullet$} 
\put(100.00,50.00){\line(-1,1){50.00}} 
\put(105.00,50.00){$2$} 
\put(95.00,47.00){$\bullet$} 
\put(100.00,50.00){\line(0,-1){50.00}} 
\put(105.00,0.00){$3$} 
\put(95.00,-3.00){$\bullet$} 

\put(0.00,0.00){\line(1,0){100.00}} 
\put(-13.00,0.00){$4$} 
\put(-1.00,-3.00){$\bullet$} 

\end{picture}
\caption{A graph with $5$ vertices.}
\label{fig:chip-game}
\end{figure}
This graph has incidence matrix

\[
D = 
\left(
\begin{array}{ccccc}
-1 & -1 & 0 & 0 & 0 \\
0 & -1 & -1 & 0 & 0 \\
1 & 0 & -1 & -1 & 0 \\
0 & 0 & 0 & -1 & -1 \\
0 & 0 & 0 &  0  &  1  \\
\end{array}
\right),
\]
and Laplacian

\[
L = D\cdot \, ^tD = 
\left(
\begin{array}{ccccc}
2 & -1 & 0-1 & 0 & 0 \\
-1 & 2 & -1 & 0 & 0 \\
-1 & -1 & 3 & -1 & 0 \\
0 & 0 & -1 & 2 & -1 \\
0 & 0 & 0 &  -1  &  1  \\
\end{array}
\right).
\]
Suppose the initial configuration is 
$s = (3,1,0,1,1)$, i.e., 

\begin{itemize}
\item
player $0$ has $3$ dollars,
\item
player $1$ has $1$ dollar,
\item
player $2$ has nothing,
\item
player $3$ has $1$ dollar,
\item
player $4$ has $1$ dollar.
\end{itemize}
Notice player $0$ is active. If we fire $0$ then we get the new
configuration $s'=(1,2,1,1,1)$. Indeed, if 
we compute $s'=s-Lx(0)$, we get:

\[
s'=
\left(
\begin{array}{c}
3 \\
1 \\
0 \\
1 \\
1  \\
\end{array}
\right)
-
\left(
\begin{array}{ccccc}
2 & -1 & -1 & 0 & 0 \\
-1 & 2 & -1 & 0 & 0 \\
-1 & -1 & 3 & -1 & 0 \\
0 & 0 & -1 & 2 & -1 \\
0 & 0 & 0 &  -1  &  1  \\
\end{array}
\right)
\left(
\begin{array}{c}
1 \\
0 \\
0 \\
0 \\
0  \\
\end{array}
\right)
=
\left(
\begin{array}{c}
3 \\
1 \\
0 \\
1 \\
1  \\
\end{array}
\right)
-
\left(
\begin{array}{c}
2 \\
-1 \\
-1 \\
0 \\
0  \\
\end{array}
\right)
=
\left(
\begin{array}{c}
1 \\
2 \\
1 \\
1 \\
1  \\
\end{array}
\right).
\]
This can be written more concisely as

\[
(3,1,0,1,1)\overset{0}{\to} (1,2,1,1,1).
\]
We have the cycle

\[
\begin{array}{c}
(1,2,1,1,1)
\overset{1}{\to} (2,0,2,1,1)
\overset{0}{\to} (0,1,3,1,1)
\overset{2}{\to} (1,2,0,2,1)\\
\overset{3}{\to} (1,2,1,0,2)
\overset{4}{\to} (1,2,1,1,1).
\end{array}
\]



}
\end{example}


\subsubsection{Chip-firing game variants}

For simplicity, let $G=(V,E)$ be an undirected graph
with an indexed set of vertices $V=\{v_1,\dots, v_m\}$
and an indexed set of vertices $E=\{e_1,\dots, e_n\}$.

One variant (the ``sandpile model'') has a special vertex,
called ``the sink,'' which has special firing properties.
In the sandpile variant, the sink is never fired.
Another variant (the ``dollar game'') has a special vertex,
called ``the source,'' which has special firing properties.
In the dollar game variant, the source is only fired when
not other vertex is active.
We shall consider the dollar game variant here, following
Biggs \cite{Biggs2007}. 
\index{sandpile model}
\index{dollar game}

We select a distinguished vertex $q\in V$, called the
``source\footnote{Biggs humorously calls $q$ ``the government.''},''
which has a special property to be described below.
For the dollar game, a {\it configuration} is a 
function $s:V\to {\mathbb{R}}$ for which 

\[
\sum_{v\in V} s(v) = 0,\ \ \ \
\]
and $s(v)\geq 0$ for all $v\in V$ with $v\not= q$.
A vertex $v\not= q$ can be fired if and only if 
$\deg (v)\leq s(v)$ (i.e., it ``has enough chips'').
The equation (\ref{eqn:fire}) describes the 
new configuration after firing a vertex.
\index{configuration}

Here is some \sage/{\tt Python} code for determining the
configuration after firing an active vertex.

\begin{Verbatim}[fontsize=\scriptsize,fontfamily=courier,fontshape=tt,frame=single,label=\sage]

def fire(G, s, v0):
    """
    Returns the configuration after firing the active vertex v.

    INPUT:
     G - a graph
     s - a configuration (implemented as a list 
                          or a dictionary keyed on
                          the vertices of the graph)
     v - a vertex of the graph

    EXAMPLES:
        sage: A = matrix([[0,1,1,0,0],[1,0,1,0,0],[1,1,0,1,0],[0,0,0,0,1],[0,0,0,0,0]])
        sage: G = Graph(A, format = "adjacency_matrix", weighted = True)
        sage: s = {0: 3, 1: 1, 2: 0, 3: 1, 4: 1}
        sage: fire(G, s, 0)
        {0: 1, 1: 2, 2: 1, 3: 1, 4: 1}

    """
    V = G.vertices()
    j = V.index(v0)
    s1 = copy(s)
    if not(v0 in V):
        raise ValueError, "the last argument must be a vertex of the graph."
    if not(v0 in active_vertices(G, s)):
        raise ValueError, "the last argument must be an active vertex of the graph."
    degs = [G.degree(w) for w in V]
    for w in V:
         if w == v0:
             s1[v0] = s[v0] - degs[j]
         if w in G.neighbors(v0):
             s1[w] = s[w]+1
    return s1

\end{Verbatim}
We say $s:V\to {\mathbb{R}}$ is a {\it stable} configuration
if $0\leq s(v)<\deg(v)$, for all $v\not= q$.
The source vertex $q$ can only be fired when
no other vertex can be fired, that is only in the case when 
a stable configuration has been reached.
\index{configuration!stable}

Here is some \sage/{\tt Python} code for determining the stable vertices.

\begin{Verbatim}[fontsize=\scriptsize,fontfamily=courier,fontshape=tt,frame=single,label=\sage]

def stable_vertices(G, s, source = None):
    """
    Returns the list of stable vertices.

    INPUT:
     G - a graph
     s - a configuration (implemented as a list 
                          or a dictionary keyed on
                          the vertices of the graph)

    EXAMPLES:
        sage: A = matrix([[0,1,1,0,0],[1,0,1,0,0],[1,1,0,1,0],[0,0,0,0,1],[0,0,0,0,0]])
        sage: G = Graph(A, format = "adjacency_matrix", weighted = True)
        sage: s = {0: 3, 1: 1, 2: 0, 3: 1, 4: 1}
        sage: stable_vertices(G, s)

    """
    V = G.vertices()
    degs = [G.degree(v) for v in V]
    if source==None:
        stable = [v for v in V if degs[V.index(v)]>s[v]]
    else:
        stable = [v for v in V if degs[V.index(v)]>s[v] and v!=source]
    return stable

\end{Verbatim}

Suppose we are in a configuration $s_1$.
We say a sequence vertices $S=(w_1, w_2,\dots, w_k)$,
$w_i\in V$ not necessarily distinct, is
{\it legal} if, 
\index{legal firing sequence}
\begin{itemize}
\item
$w_1$ is active in configuration $s_1$,
\item
for each $i$ with $1\leq i<k$, $s_{i+1}$ is obtained
from $s_i$ by firing $w_i$ in configuration 
$s_i$, 
\item
for each $i$ with $1\leq i<k$, $w_{i+1}$ is active
in the configuration $s_{i+1}$ defined in the 
previous step,
\item
the source vertex $q$ occurs in $S$ only if
it immediately follows a stable configuration.
\end{itemize}
We call $s_1$ or $w_1$ the {\it start}
of $S$. A configuration $s$ is {\it recurrent}
\index{configuration!recurrent}
if there is a legal sequence starting at $s$ which leads back to $s$.
A configuration is {\it critical}
\index{configuration!critical}
if it recurrent and stable. 

Here is some \sage/{\tt Python} code for determining a stable
vertex resulting from a legal sequence of firings
of a given configuration $s$. I think it returns the
unique critical configuration associated to $s$ but have not
proven this.

\begin{Verbatim}[fontsize=\scriptsize,fontfamily=courier,fontshape=tt,frame=single,label=\sage]

def stabilize(G, s, source, legal_sequence = False):
    """
    Returns the stable configuration of the graph originating from
    the given configuration s. If legal_sequence = True then the
    sequence of firings is also returned. By van den Heuvel [1],
    the number of firings needed to compute a critical configuration 
    is < 3(S+2|E|)|V|^2, where S is the sum of the positive 
    weights in the configuration.

    EXAMPLES:
        sage: A = matrix([[0,1,1,0,0],[1,0,1,0,0],[1,1,0,1,0],[0,0,1,0,1],[0,0,0,1,0]])
        sage: G = Graph(A, format="weighted_adjacency_matrix")
        sage: s = {0: 3, 1: 1, 2: 0, 3: 1, 4: -5}
        sage: stabilize(G, s, 4)
        {0: 0, 1: 1, 2: 2, 3: 1, 4: -4}

    REFERENCES:
        [1] J. van den Heuvel, "Algorithmic aspects of a chip-firing
            game," preprint.
    """
    V = G.vertices()
    E = G.edges()
    fire_number = 3*len(V)^2*(sum([s[v] for v in V if s[v]>0])+2*len(E))+len(V)
    if legal_sequence:
        seq = []
    stab = []
    ac = active_vertices(G,s)
    for i in range(fire_number):
        if len(ac)>0:
            s = fire(G,s,ac[0])
            if legal_sequence:
                seq.append(ac[0])
        else:
            stab.append(s)
            break
        ac = active_vertices(G,s)
    if len(stab)==0:
        raise ValueError, "No stable configuration found."
    if legal_sequence:
        return stab[0], seq
    else:
        return stab[0]

\end{Verbatim}

The
incidence matrix $D$ and its transpose $^tD$ can be regarded as
homomorphisms

\[
D : C^1(G,{\mathbb{Z}})\to C^0(G,{\mathbb{Z}}) \ \ \ \ {\rm and}\ \ \ \ \ 
^tD : C^0(G,{\mathbb{Z}})\to C^1(G,{\mathbb{Z}}).
\]
We can also regard the Laplacian $L = D\cdot \, ^tD$ as a homomorphism 
\index{Laplacian}
$C^0(G,{\mathbb{Z}})\to C^0(G,{\mathbb{Z}})$. Denote by
$\sigma : C^0(G,{\mathbb{Z}})\to {\mathbb{Z}}$ the homomorphism defined by 

\[
\sigma(f) = \sum_{v\in V} f(v).
\]
Denote by $K(G)$ the set of {\it critical configurations} 
on a graph $G$.
\index{critical configurations} 

\begin{lemma} (Biggs \cite{Biggs2007})
{\rm
The set $K(G)$ of critical configurations on a connected graph $G$ is in
bijective correspondence with the abelian group 
${\rm Ker} (\sigma)/{\rm Im}( Q)$.
}
\end{lemma}
\index{critical group}

If you accept this lemma (which we do not prove here) then
you must believe that there is a bijection
$f:K(G)\to {\rm Ker} (\sigma)/{\rm Im}( Q)$. Now, a group
operation $\bullet $ on $K(G)$ an be defined by

\[
a\bullet b = f^{-1}(f(a)+f(b)),
\]
for all $a,b\in {\rm Ker} (\sigma)/{\rm Im}( Q)$.

\begin{example}
{\rm
Consider again the graph

\vskip .2in
\begin{figure}[h!]

% this was done "manually"
\unitlength=0.620000pt
\begin{picture}(250.00,100.00)(-250.00,0.00)

\thinlines
\put(0.00,50.00){\line(1,0){100.00}} 
\put(-1.00,47.00){$\bullet$} 
\put(-13.00,50.00){$0$} 
\put(0.00,50.00){\line(1,1){50.00}} 
\put(50.00,112.00){$1$} 
\put(45.00,97.00){$\bullet$} 
\put(100.00,50.00){\line(-1,1){50.00}} 
\put(105.00,50.00){$2$} 
\put(95.00,47.00){$\bullet$} 
\put(100.00,50.00){\line(0,-1){50.00}} 
\put(105.00,0.00){$3$} 
\put(95.00,-3.00){$\bullet$} 

\put(0.00,0.00){\line(1,0){100.00}} 
\put(-15.00,0.00){$4$} 
\put(-4.00,-4.00){$\circ$} 

\end{picture}
\caption{A graph with $5$ vertices.}
\label{fig:chip-game2}
\end{figure}
This graph has incidence matrix

\[
D = 
\left(
\begin{array}{ccccc}
-1 & -1 & 0 & 0 & 0 \\
0 & -1 & -1 & 0 & 0 \\
1 & 0 & -1 & -1 & 0 \\
0 & 0 & 0 & -1 & -1 \\
0 & 0 & 0 &  0  &  1  \\
\end{array}
\right),
\]
and Laplacian

\[
L = D\cdot \, ^tD = 
\left(
\begin{array}{ccccc}
2 & -1 & 0-1 & 0 & 0 \\
-1 & 2 & -1 & 0 & 0 \\
-1 & -1 & 3 & -1 & 0 \\
0 & 0 & -1 & 2 & -1 \\
0 & 0 & 0 &  -1  &  1  \\
\end{array}
\right).
\]
Suppose the initial configuration is 
$s = (3,1,0,1,-5)$, i.e., 

\begin{itemize}
\item
player $0$ has $3$ dollars,
\item
player $1$ has $1$ dollar,
\item
player $2$ has nothing,
\item
player $3$ has $1$ dollar,
\item
player $4$ is the source vertex $q$.
\end{itemize}
The legal sequence $(0,1,0,2,1,0,3,2,1,0)$ leads to
the stable configuration $(0,1,2,1,-4)$. If $q$ is fired
then the configuration $(0,1,2,2,-5)$ is achieved.
This is recurrent since it is contained in the 
cyclic legal sequence

\[
\begin{array}{c}
(0,1,2,2,-5)
\overset{3}{\to} (0,1,3,0,-4)
\overset{2}{\to} (1,2,0,1,-4)\\
\overset{1}{\to} (2,0,1,1,-4)
\overset{0}{\to} (0,1,2,1,-4)
\overset{q}{\to} (0,1,2,2,-5).
\end{array}
\]
In particular, the configuration $(0,1,2,1,-4)$ is 
also recurrent. Since it is both stable and recurrent, it
is critical.
}
\end{example}

The following result is of basic importance but
I'm not sure who proved it first. It is quoted in many of the papers
on this topic in one form or another.

\begin{theorem} (Biggs \cite{Biggs1999}, Theorem 3.8)
{\rm
If $s$ is an configuration and $G$ is connected
then there is a unique critical configuration $s'$
which can be obtained by a sequence of legal firings 
for starting at $s$.
}
\end{theorem}

The map defined by the above theorem is
denoted

\[
\gamma :C^0(G,{\mathbb{R}})\to K(G).
\]
Another way to define multiplication $\bullet$ on
on $K(G)$ is 

\[
\gamma(s_1)\bullet \gamma(s_2)
=\gamma(s_1+s_2),
\]
where $s_1+s_2$ is computed using addition on
$C^0(G,{\mathbb{R}})$. According to Perkinson \cite{PerkinsonEtAl2009},
\index{critical group}
Theorem 2.16, the critical group satisfies the
following isomorphism:

\[
K(G)\cong {\mathbb{Z}}^{m-1}/{\mathcal{L}},
\]
where ${\mathcal{L}}$ is the integer lattice 
generated by the columns of the reduced
Laplacian matrix\footnote{The {\it reduced Laplacian} matrix
is obtained from the Laplacian matrix by removing 
the row and column associated to the source 
vertex.}.
\index{reduced Laplacian}
\index{Laplacian!reduced}

If $s$ is a configuration then we define

\[
{\rm wt}(s) = \sum_{v\in V, v\not= q} s(q)
\]
to be the {\it weight} of the configuration.
\index{configuration!weight}
The {\it level} of the configuration is defined by
\index{configuration!level}

\[
{\rm level}(s) = {\rm wt}(s)-|E|+\deg(q).
\]

\begin{lemma}
(Merino \cite{Morino1999})
{\rm
If $s$ is a critical configuration then

\[
0\leq {\rm level}(s)\leq |E|-|V|+1.
\]
}
\end{lemma}

This is proven in Theorem 3.4.5 in \cite{Morino1999}.
What is also proven in \cite{Morino1999} is a statement
whcih computes the number of critical
configurations of a given level in terms of the
Tutte polynomial of the associated graph.



%%%%%%%%%%%%%%%%%%%%%%%%%%%%%%%%%%%%%%%%%%%%%%%%%%%%%%%%%%%%%%%%%%%%%%%%%%%




\section{Ford-Fulkerson theorem}

The Ford-Fulkerson Theorem, or ``Max-flow/Min-cut Theorem,''
was proven by P. Elias, A. Feinstein, and C.E. Shannon in 1956, and,
independently, by L.R. Ford, Jr.  and D.R. Fulkerson in the same year.
So it should be called the
``Elias-Feinstein-Ford-Fulkerson-Shannon Theorem,''
to be precise about the authorship.

To explain the meaning of this theorem, we need to introduce some
notation and  terminology.

Consider an edge-weighted simple
digraph $G=(V,E,i,h)$ without negative weight
cycles. Here $E\subset V^{(2)}$,
$i$ is an incidence function as in~\eqref{eqn:edge-incidence}, which
we regard as the identity function, and $h$ is an
orientation function as in~\eqref{eqn:edge-orientation}.
Let $G$ be a {\it network},
\index{network}
with two distinguished vertices, the ``source'' and the ``sink.''
Let $s$ and $t$ denote the source and the sink of $G$, respectively.
The {\it capacity} (or {\it edge capacity})
\index{capacity}
\index{edge!capacity})
is a mapping $c: E \to {\mathbf{R}}$, denoted by $c_{uv}$
or $c(u,v)$, for $(u,v)\in E$ and $h(e)= u$.
If $(u,v)\in E$ and $h(e)= v$
then we set, by convention, $c(v,u)=-c(u,v)$.
Thinking of a graph as a network of pipes (representing the edges)
transporting water with various junctions (representing vertices),
the capacity function represents the maximum amount
of ``flow'' that can pass through an edge.

A {\it flow}
\index{flow}
is a mapping $f: E \to {\mathbf{R}}$, denoted by $f_{uv}$ or
$f(u,v)$, subject to the following two constraints:
\begin{itemize}
\item
$f(u,v)\leq c(u,v)$, for each $(u,v) \in V$ (the ``capacity constraint''),
\item
$\sum_{u\in V,\ (u,v)\in E} f(u,v) = \sum_{u\in V,\ (v, u)\in E} f(v, u)$ ,
for each $v\in V$ (conservation of flows).
\end{itemize}
An edge $(u,v) \in E$ is {\it $f$-saturated}
\index{$f$-saturated}
\index{saturated edge}
if $f(u,v)=c(u,v)$.
An edge $(u,v) \in E$ is {\it $f$-zero} if $f(u,v)=0$.
\index{$f$-zero}
A path with available capacity is called an ``augmenting path.''
More precisely, a directed path form $s$ to $t$ is
{\it $f$-augmenting}, or
\index{$f$-augmenting}
\index{$f$-unsaturated}
\index{augmenting path}
$f$-unsaturated, if no forward
edge is $f$-saturated and no backward edge is $f$-zero.

The {\it value of the flow} is defined by

\[
| f | = \sum_{v\in V}f(s,v)-\sum_{v\in V}f(v,s),
\]
where $s$ is the source.
It represents the amount of flow passing from the source to the sink.
\index{flow!value}
\index{value of flow}
The {\it maximum flow problem} is to maximize $| f |$, that is, to route as
much flow as possible from $s$ to $t$.
\index{maximum flow problem}

\begin{example}
{\rm
Consider the digraph having adjacency matrix

\[
\left(\begin{array}{cccccc}
0 & 1 & 1 & 0 & 0 & 0 \\
-1 & 0 & -1 & 1 & 0 & 1 \\
-1 & 1 & 0 & 0 & 1 & 0 \\
0 & -1 & 0 & 0 & 0 & 1 \\
0 & 0 & -1 & 0 & 0 & 1 \\
0 & -1 & 0 & -1 & -1 & 0
\end{array}\right),
\]
depicted in Figure \ref{fig:network_flows:digraph_flow}.

\begin{figure}[!htbp]
\centering
\includegraphics{image/network-flows/digraph-flow}
\caption{A digraph with $6$ vertices.}
\label{fig:network_flows:digraph_flow}
\end{figure}
Suppose that each edge has capacity $1$.
A maximum flow $f$ is obtained by taking a flow value
of $1$ along each edge of the path

\[
p_1:(0,1),(1,5),
\]
and a flow value
of $1$ along each edge of the path

\[
p_2:(0,2),(2,4),(4,5).
\]
The maximum value of the flow in this case is $|f|=2$.

This graph can be created in Sage using the commands

\begin{lstlisting}
sage: B = matrix([[0,1,1,0,0,0],[0,0,0,1,0,1],[0,1,0,0,1,0],[0,0,0,0,0,1],[0,0,0,0,0,1],[0,0,0,0,0,0]])
sage: H = DiGraph(B, format = "adjacency_matrix", weighted=True)
\end{lstlisting}

\noindent
Type {\tt H.show(edge\_labels=True)} if you want to see the graph with
the capacities labeling the edges.


}
\end{example}


Given a capacitated digraph with capacity $c$ and flow $f$,
we define the {\it residual digraph} $G_f=(V,E)$ to be the
digraph with capacity $c_f(u,v) = c(u,v) - f(u,v)$ and no flow.
In other words, $G_f$ is the same graph but it has a different
capacity $c_f$ and flow $0$.
\index{residual digraph}
This is also called a {\it residual network}.
\index{residual network}

Define an {\it $s-t$ cut} in our capacitated digraph $G$
to be a partition $C = (S,T)$ of $V$ such that
$s \in S$ and $t\in T$.
Recall the cut-set of $C$ is the set

\[
\{(u,v)\in E\ |\ u\in S, v\in T\}.
\]

\begin{lemma}
\label{lemma:flow=0}
{\rm
Let $G = (V, E)$ be a capacitated digraph with
capacity $c: E \to {\mathbf{R}}$, and let
$s$ and $t$ denote the source and the sink of $G$, respectively.
If $C$ is an $s-t$ cut and if
the edges in the cut-set of $C$ are removed, then $| f | = 0$.
}
\end{lemma}

\begin{exercise}
Prove Lemma \ref{lemma:flow=0}.
\end{exercise}

The {\it capacity of an $s-t$ cut}
\index{capacity!cut}
$C = (S,T)$ is defined by

\[
c(S,T) = \sum_{(s,t)\in (S,T)} c(u,v).
\]
The {\it minimum cut problem}
\index{minimum cut problem}
is to minimize the amount of capacity of an $s-t$ cut.

The following theorem is due to P. Elias, A. Feinstein, L.R. Ford,
Jr.,  D.R. Fulkerson, C.E. Shannon.

\begin{theorem}
(max-flow min-cut theorem)
{\rm
The maximum value of an $s$-$t$ flow is equal to the minimum capacity of
an $s$-$t$ cut.
}
\end{theorem}
\index{max-flow min-cut theorem}

The intuitive explanation of this result is as follows.

Suppose that $G=(V,E)$ is a graph where each edge has capacity $1$.
Let $s\in V$ be the source and $t\in V$ be the sink.
The maximum flow from $s$ to $t$ is the maximum number of
independent paths from $s$ to $t$.
Denote this maximum flow by $m$.
Each $s$-$t$ cut must intersect each $s$-$t$ path at least once.
In fact, if $S$ is a minimal $s$-$t$ cut then for each
edge $e$ in $S$ there is an $s$-$t$ path containing
$e$. Therefore, $|S|\leq e$.

On the other hand, since each edge has unit capacity,
the maximum flow value can't exceed the number of
edges separating $s$ from $t$, so $m\leq |S|$.


\begin{remark}
Although the notion of an independent path is important
for the network-theoretic proof of Menger's theorem
(which we view as a corollary to the Ford-Fulkerson
theorem on network flows on networks having
capacity $1$ on all edges), its significance is less
important for networks having arbitrary capacities.
One must use caution in generalizing the above
intuitive argument to establish a rigorous proof
of the general version of the MFMC theorem.
\end{remark}

\begin{remark}
\label{remark:GMCMF}
{\rm
This theorem can be generalized as follows.
In addition to edge capacity, suppose there is capacity at each {\it vertex},
that is, a mapping $c: V \to {\mathbf{R}}$, denoted by
$v \mapsto c(v)$, such that the flow $f$ has to
satisfy not only the capacity constraint and the conservation of flows,
but also the vertex capacity constraint

\[
 \sum_{w\in V} f(w,v) \leq c(v),
\]
for each $v \in V-\{s,t\}$.
Define an {\it $s-t$ cut} to be the set of vertices and edges such
that for any path from $s$ to $t$, the path contains a member of the cut.
In this case, the capacity of the cut is the sum the capacity of each
edge {\it and} vertex in it.
In this new definition, the {\it generalized max-flow min-cut theorem}
\index{max-flow min-cut theorem!generalized}
states that the maximum value of an $s-t$ flow is equal to the minimum
capacity of an $s-t$ cut..
}
\end{remark}

The idea behind the Ford-Fulkerson algorithm is very simple: As long as
there is a path from the source to the sink, with
available capacity on all edges in the
path, we send as much flow as we can alone along each
of these paths. This is done inductively, one path at a time.

\begin{algorithm}[!htbp]
%%%%%%%%%%%%%%%%%%%%%%%%%%%%%%%%%%%%%%%%%%%%%%%%%%%%%%%%%%%%%%%%%%%%%%%%%%%
%% This file is part of the book
%%
%% Algorithmic Graph Theory
%% http://code.google.com/p/graph-theory-algorithms-book/
%%
%% Copyright (C) 2009, 2010 Minh Van Nguyen <nguyenminh2@gmail.com>
%%
%% See the file COPYING for copying conditions.
%%%%%%%%%%%%%%%%%%%%%%%%%%%%%%%%%%%%%%%%%%%%%%%%%%%%%%%%%%%%%%%%%%%%%%%%%%%

\DontPrintSemicolon
\SetAlgoNoLine
%%
%% data section
\SetKwInOut{Input}{Input}
\SetKwInOut{Output}{Output}
%%
%% input/output
\Input{Graph $G=(V,E)$ with flow capacity $c$, source $s$, and sink $t$.}
\Output{A flow $f$ from $s$ to $t$ which is a maximum for all edges in $E$.}
\BlankLine
%%
%% algorithm body
Initialize $f(u,v)=0$, for all edges $(u,v)\in E$\;

While there is a path $p$ from $s$ to $t$ in $G_f$, such that
$c_f(e) > 0$, for all edges $e\in E$:\;
\quad       Find $c_f(p) = min\{ c_f(u,v)\ |\ (u,v) \in p\}$,\;
\quad       For each edge $(u,v) \in $:\;
\quad   \quad        $f(u,v) = f(u,v) + c_f(p)$\;
\quad   \quad        $f(v,u) = f(v,u) - c_f(p)$\;

\caption{Ford-Fulkerson algorithm.}
\label{alg:distance-connectivity:ford-fulkerson}
\end{algorithm}

To prove the max-flow/min-cut theorem we will use the following lemma.

\begin{lemma}
{\rm
Let $G=(V,E)$ be a directed graph with edge
capacity $c: E \to {\mathbf{Z}}$,
a source $s\in V$, and a sink $t\in V$.
A flow $f: E \to {\mathbf{Z}}$ is a maximum flow if
and only if there is no $f$-augmenting path in the graph.
}
\end{lemma}

In other words, a flow $f$ in a
capacitated network is a maximum flow if and only if
there is no $f$-augmenting path in the network.

\begin{proof}[Solution]
One direction is easy. Suppose that the flow is a maximum.
If there is an $f$-augmenting path then the
current flow can be increased using that path, so the flow would not
be a maximum. This contradiction proves the ``only if'' direction.

Now, suppose there is no $f$-augmenting path in the network.
Let $S$ be the set of vertices $v$ such that there is an $f$-unsaturated path from
the source $s$ to $v$. We know $s\in S$ and (by hypothesis)
$t\notin S$. Thus there is a cut of the form $(S,T)$ in the network.
Let $e=(v,w)$ be any edge in this cut, $v\in S$ and $w\in T$. Since
there is no $f$-unsaturated path from $s$ to $w$,
$e$ is $f$-saturated. Likewise, any edge in the cut
$(T,S)$ is $f$-zero. Therefore, the current flow value is equal to the
capacity of the cut $(S,T)$. Therefore, the current flow is a maximum.
\end{proof}

We can now prove the max-flow/min-cut theorem.

\begin{proof}[Solution]
Let $f$ be a maximum flow.
If

\[
S = \{v\in V\ |\ {\rm there\ exists\ an\ }f-{\rm saturated\ path\
  from\ }s\ {\rm to\ }v\},
\]
then by the previous lemma, $S\not= V$.
Since $T=V-S$ is non-empty, there is a cut $C=(S,T)$.
Each edge of this cut $C$ in the capacitated network $G$ is
$f$-saturated.

\end{proof}

Here is some Python code\footnote{Please see
\url{http://en.wikipedia.org/wiki/Ford-Fulkerson_algorithm}.}
which implements this. The class {\tt FlowNetwork} is basically a Sage
Graph class with edge weights and an extra data structure representing
the flow on the graph.

%%%%%%%%%%%%%%%%%%%%%%%%%%%%%%%%%%%%%%%%%%%%%%%%%%
% TODO: Translate this into Sage
%% easiler said than done, since the flow "record" is missing from
%% Sage. One way: create new functions "find_path" and "max_flow"
%% similar to the ones below but with a new argument
%% "flow" which is a function on the edges of G. Maybe there
%% is a better way??
%%%%%%%%%%%%%%%%%%%%%%%%%%%%%%%%%%%%%%%%%%%%%%%%%%
\begin{lstlisting}
class Edge:
    def __init__(self,U,V,w):
        self.source = U
        self.to = V
        self.capacity = w
    def __repr__(self):
        return str(self.source) + "->" + str(self.to) + " : " + str(self.capacity)

class FlowNetwork(object):
    """
    This is a graph structure with edge capacities.

    EXAMPLES:
        g=FlowNetwork()
        map(g.add_vertex, ['s','o','p','q','r','t'])
        g.add_edge('s','o',3)
        g.add_edge('s','p',3)
        g.add_edge('o','p',2)
        g.add_edge('o','q',3)
        g.add_edge('p','r',2)
        g.add_edge('r','t',3)
        g.add_edge('q','r',4)
        g.add_edge('q','t',2)
        print g.max_flow('s','t')
    """
    def __init__(self):
        self.adj, self.flow, = {},{}

    def add_vertex(self, vertex):
        self.adj[vertex] = []

    def get_edges(self, v):
        return self.adj[v]

    def add_edge(self, u,v,w=0):
        assert(u != v)
        edge = Edge(u,v,w)
        redge = Edge(v,u,0)
        edge.redge = redge
        redge.redge = edge
        self.adj[u].append(edge)
        self.adj[v].append(redge)
        self.flow[edge] = self.flow[redge] = 0

    def find_path(self, source, sink, path):
        if source == sink:
            return path
        for edge in self.get_edges(source):
            residual = edge.capacity - self.flow[edge]
            if residual > 0 and not (edge,residual) in path:
                result = self.find_path(edge.to, sink, path + [ (edge,residual) ])
                if result != None:
                    return result

    def max_flow(self, source, sink):
        path = self.find_path(source, sink, [])
        while path != None:
            flow = min(res for edge,res in path)
            for edge,res in path:
                self.flow[edge] += flow
                self.flow[edge.redge] -= flow
            path = self.find_path(source, sink, [])
        return sum(self.flow[edge] for edge in self.get_edges(source))
\end{lstlisting}


%%%%%%%%%%%%%%%%%%%%%%%%%%%%%%%%%%%%%%%%%%%%%%%%%%%%%%%%%%%%%%%%%%%%%%%%%%%

\section{Edmonds and Karp's algorithm}

The objective of this section is to prove Edmond and Karp's algorithm
for the maximum flow-minimum cut problem with polynomial
complexity.


%%%%%%%%%%%%%%%%%%%%%%%%%%%%%%%%%%%%%%%%%%%%%%%%%%%%%%%%%%%%%%%%%%%%%%%%%%%

\section{Goldberg and Tarjan's algorithm}

The objective of this section is to prove Goldberg and Tarjan's
algorithm for finding maximal flows with polynomial complexity.
