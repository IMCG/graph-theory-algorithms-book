%%%%%%%%%%%%%%%%%%%%%%%%%%%%%%%%%%%%%%%%%%%%%%%%%%%%%%%%%%%%%%%%%%%%%%%%%%%
%% This file is part of the book
%%
%% Algorithmic Graph Theory
%% http://code.google.com/p/graph-theory-algorithms-book/
%%
%% Copyright (C) 2009, 2010 Minh Van Nguyen <nguyenminh2@gmail.com>
%% Copyright (C) 2010 Nathann Cohen <nathann.cohen@gmail.com>
%%
%% See the file COPYING for copying conditions.
%%%%%%%%%%%%%%%%%%%%%%%%%%%%%%%%%%%%%%%%%%%%%%%%%%%%%%%%%%%%%%%%%%%%%%%%%%%

\chapter{Planar Graphs}
\label{chap:planar_graphs}

A {\it planar graph} is a graph that can be drawn on a sheet of paper without any overlapping between its edges.

It is a property of many ``natural'' graphs drawn on the earth's surface, like for instance the graph of roads, or the graph of internet fibers. It is also a necessary property of graphs we want to build, like VLSI layouts.

Of course, the property of being planar does not prevent one from finding a drawing with many overlapping between edges, as this property only asserts that there exists a drawing (or {\it embedding}) of the graph avoiding it. Planarity can be characterized in many different ways, one of the most satiating being Kuratowski's theorem.

See chapter~9 of Gross and Yellen~\cite{GrossYellen1999}.


%%%%%%%%%%%%%%%%%%%%%%%%%%%%%%%%%%%%%%%%%%%%%%%%%%%%%%%%%%%%%%%%%%%%%%%%%%%

\section{Planarity and Euler's Formula}

\begin{itemize}
\item planarity, non-planarity, planar and plane graphs

\item crossing numbers
\end{itemize}

\begin{figure}[!htbp]
\centering
%%%%%%%%%%%%%%%%%%%%%%%%%%%%%%%%%%%%%%%%%%%%%%%%%%%%%%%%%%%%%%%%%%%%%%%%%%%
%% This file is part of the book
%%
%% Algorithmic Graph Theory
%% http://code.google.com/p/graph-theory-algorithms-book/
%%
%% Copyright (C) 2009, 2010, 2011 Minh Van Nguyen <nguyenminh2@gmail.com>
%%
%% See the file COPYING for copying conditions.
%%%%%%%%%%%%%%%%%%%%%%%%%%%%%%%%%%%%%%%%%%%%%%%%%%%%%%%%%%%%%%%%%%%%%%%%%%%

\subfigure[Errera graph.]{
\begin{tikzpicture}
[lineDecorate/.style={-,thick},%
  nodeDecorate/.style={shape=circle,inner sep=2pt,draw,thick},%
  scale=2]
%% nodes or vertices
\foreach \nodename/\x/\y in {
  0/2/0, 1/1.4142/1.4142, 2/0/2, 3/-1.4142/1.4142, 4/-2/0,
  5/-1.4142/-1.4142, 6/0/-2, 7/1.4142/-1.4142, 8/0.25/0, 9/-0.25/0.7071,
  10/-1.1642/0.7071, 11/-1.4142/0, 12/-1.1642/-0.7071,
  13/-0.25/-0.7071, 14/1.6642/0.7071, 15/1.1642/0, 16/1.6642/-0.7071}
{
  \node (\nodename) at (\x,\y) [nodeDecorate] {};
}
%% edges or lines
\tikzstyle{EdgeStyle}=[-,thick]
\foreach \startnode/\endnode/\bend in {
  0/1/bend right, 0/7/bend left, 0/14/bend left=0, 0/15/bend left=0,
  0/16/bend left=0, 1/2/bend right, 1/9/bend left=0, 1/14/bend left=0,
  1/15/bend left=0, 2/3/bend right, 2/8/bend left=0, 2/9/bend left=0,
  2/10/bend left=0, 2/14/bend left=0, 3/4/bend right, 3/9/bend left=0,
  3/10/bend left=0, 3/11/bend left=0, 4/5/bend right, 4/10/bend
  left=0, 4/11/bend left=0, 4/12/bend left=0, 5/6/bend right,
  5/11/bend left=0, 5/12/bend left=0, 5/13/bend left=0, 6/7/bend
  right, 6/8/bend left=0, 6/12/bend left=0, 6/13/bend left=0,
  6/16/bend left=0, 7/13/bend left=0, 7/15/bend left=0, 7/16/bend left=0,
  8/10/bend left=0, 8/12/bend left=0, 8/14/bend left=0, 8/16/bend left=0,
  9/11/bend left=0, 9/13/bend left=0, 9/15/bend left=0, 10/12/bend left=0,
  11/13/bend left=0, 13/15/bend left=0, 14/16/bend left=0}
{
  \Edge[style=\bend](\startnode)(\endnode)
}
\end{tikzpicture}
}
%%
%%
\subfigure[Planar representation.]{
\begin{tikzpicture}
[lineDecorate/.style={-,thick},%
  nodeDecorate/.style={shape=circle,inner sep=2pt,draw,thick},%
  scale=2]
%% nodes or vertices
\foreach \nodename/\x/\y in {
  0/3.4641/-2, 1/0/4, 2/-3.4641/-2, 3/0/3, 4/-1.73205/1, 5/-1.732/-1,
  6/0/-1.5, 7/1.732/-1, 8/1.73205/1, 9/0.5/2.3333, 10/-0.5/2.3333,
  11/-0.683/1, 12/-0.866/-0.3333, 13/0.866/-0.3333, 14/0.683/1,
  15/0/1.66665, 16/0/0.33335}
{
  \node (\nodename) at (\x,\y) [nodeDecorate] {};
}
%% edges or lines
\tikzstyle{EdgeStyle}=[-,thick]
\foreach \startnode/\endnode/\bend in {
  0/1/bend right, 0/6/bend left=25, 0/7/bend left=0, 0/8/bend right=25,
  1/2/bend right, 1/3/bend left=0, 1/4/bend right=25, 1/8/bend left=25,
  2/0/bend right, 2/5/bend left=0, 2/4/bend left=25, 2/6/bend right=25,
  3/4/bend right=25, 3/8/bend left=25, 3/9/bend left=25, 3/10/bend right=25,
  4/10/bend left=25, 5/4/bend left=25, 5/6/bend right=25, 5/10/bend left=25,
  5/11/bend left=25, 5/12/bend right=25, 6/12/bend left=25, 6/13/bend right=25,
  7/6/bend left=25, 7/8/bend right=25, 7/9/bend right=25, 7/14/bend right=25,
  7/13/bend left=25, 8/9/bend right=25, 9/10/bend right=25, 9/14/bend left=25,
  9/15/bend left=25, 10/11/bend right=25, 10/15/bend right=25,
  11/12/bend right=25, 11/15/bend left=25, 11/16/bend right=25,
  12/13/bend right=25, 12/16/bend right=25, 13/14/bend right=25,
  13/16/bend left=25, 14/15/bend right=25, 14/16/bend left=25,
  15/16/bend left=0}
{
  \Edge[style=\bend](\startnode)(\endnode)
}
\end{tikzpicture}
}

\caption{The Errera graph is planar.}
\label{fig:planar_graphs:Errera_graph}
\end{figure}

\begin{theorem}
The complete bipartite graph $K_{3,n}$ is non-planar for $n \geq 3$.
\end{theorem}

\begin{theorem}
\textbf{Euler's Formula.}
Let $G$ be a connected plane graph having $n$ vertices, $e$ edges and
$f$ faces. Then $n - e + f = 2$.
\end{theorem}


%%%%%%%%%%%%%%%%%%%%%%%%%%%%%%%%%%%%%%%%%%%%%%%%%%%%%%%%%%%%%%%%%%%%%%%%%%%

\section{Kuratowski's Theorem}

\begin{itemize}
\item Kuratowski graphs
\end{itemize}

The objective of this section is to prove the following theorem.

\begin{theorem}\cite{kuratowski1930}
\label{thm:kuratowski}
{\bf Kuratowski's Theorem.}
A graph is planar if and only if it contains no subgraph homeomorphic
to a subdivision of $K_5$ or $K_{3,3}$.
\end{theorem}

{\bf Graph Minors :} The reader may find interesting to notice that the previous result, first proved in 1930 as purely topological (there is no mention of graphs in Kuratowski's original paper), can be seen as a very special case of the Graph Minor Theorem (Thm\ref{thm:introduction:graph_minor}).

It can easily be seen that if a graph $G$ is planar, any of its subgraph is also planar. Besides, planarity is still preserved under edge contraction. These two facts mean together that any minor of a planar graph is still planar graph, which makes of planarity a {\it minor-closed} property. If we let $\bar{\mathcal P}$ denote the poset of all non-planar graph, ordered with the minor partial order, we can now consider the set $\bar{\mathcal P}_{min}$ of its minimal elements which, by the Graph Minor Theorem, is a finite set.

Actually, Kuratowski's theorem asserts that $\bar{\mathcal P}_{min}=\{K_5,K_{3,3}\}$.


%%%%%%%%%%%%%%%%%%%%%%%%%%%%%%%%%%%%%%%%%%%%%%%%%%%%%%%%%%%%%%%%%%%%%%%%%%%

\section{Planarity algorithms}

\begin{itemize}
\item planarity testing for $2$-connected graphs

\item planarity testing algorithm of Hopcroft and
  Tarjan~\cite{HopcroftTarjan1974}

\item planarity testing algorithm of Boyer and
  Myrvold~\cite{BoyerMyrvold2004}
\end{itemize}
