%%-----------------------------------------------------------------------%%
%%--- Distance and Connectivity -----------------------------------------%%

\chapter{Distance and Connectivity}
\label{chap:distance_connectivity}


%%-----------------------------------------------------------------------%%
%%--- Paths and distance ------------------------------------------------%%

\section{Paths and distance}

\begin{itemize}
\item distance and metrics

\item distance matrix

\item eccentricity, center, radius, diameter

\item trees: distance, center, centroid

\item distance in self-complementary graphs
\end{itemize}

\subsection{Distance and metrics on graphs}

Consider an edge-weighted graph $G=(V,E)$ without negative weight
cycles. Let ${\rm wt}:E\to {\mathbb{R}}$ be the weight function.
(if $G$ is not provided with a weight function on the edges,
then assume that ${\rm wt}(e)=1$, for all $e\in E$.)
If $v_1,v_2\in V$ are two vertices and $P=(e_1,\dots,e_m)$ is
a path from $v_1$ to $v_2$ (so $v_1$ is incident to $e_1$
nd $v_2$ is incident to $e_m$), we define the {\it weight of $P$}
to be the sum of the weights of the edges in $P$:

\[
{\rm wt}(P)=\sum_{i=1}^m {\rm wt}(e_i).
\]
The {\it distance function} $\partial:V\times V\to  {\mathbb{R}}\cup
\{\infty\}$ on 
\index{distance function}
$G$ is defined by $\partial(v_1,v_2) = \infty$,
if $v_1$ and $v_2$ lie in distinct connected
components of $G$, and by

\[
\partial(v_1,v_2) = 
\min_P {\rm wt}(P),
\]
otherwise, where the minimum is taken over all paths $P$
from $v_1$ to $v_2$ (this minimum exists since 
$G$ has no negative weight cycles. 

This distance function is not in general a metric
(that is, the triagle inequality is not true in general).
However, when the distance function is a 
metric then $G$ is called a {\it metric graph}.
\index{graph!metric}
The theory of metric graphs, thanks to their close
connection with tropical curves, is a very active
research area at the present (see the work of 
M. Baker et al, e.g., \cite{BakerFaber2006}).

\begin{exercise}
Show that the vertex set $V$ (of an undirected unweighted simple graph) 
and the distance function form
a metric space, if and only if the graph is connected.
\end{exercise}

The {\it eccentricity} $\epsilon:V\to  {\mathbb{R}}$ is defined as follows:
$\epsilon (v)$ (for a vertex $v$) is the greatest distance between $v$
and any other vertex in $G$.
\index{eccentricity}

The {\it diameter} of $G$, $\partial(G)$, is the maximum 
eccentricity of any vertex in the graph. To compute
$\partial(G)$, first find the shortest path (see Chapter
\ref{chap:graph_algorithms}) between each pair of 
vertices. The maximum weight of any of
these paths is the diameter of the graph:

\[
\partial(G) = \max_{v_1,v_2\in V} \partial(v_1,v_2).
\]
\index{diameter}

To compute the diameter of a graph, use the Floyd-Roy-Warshall
algorithm to compute the shortest distance between any
two pairs of vertices. The maximum of these distances is the
diameter.


\subsection{Distance matrix}

There are two uses of this term in the literature.

Consider an edge-weighted graph $G=(V,E)$ without negative weight
cycles with distance function $\partial$. Let $d=\partial(G)$
denote the diameter of $G$ and index the
set of vertices in some arbitrary but fixed way,
$V=\{v_1,v_2,\dots, v_n\}$. 
The {\it distance matrices} of $G$ are a sequence of $n\times n$ matrices
$\{A_1,\dots, A_d)$, where

\[
(A_k)_{ij} = 
\left\{
\begin{array}{ll}
1,& {\rm if}\ \partial(v_i,v_j)=k,\\
0,& {\rm otherwise}.
\end{array}
\right.
\]
In particular, $A_1$ is the usual adjacency matrix $A$.

To compute the distance matrices of a graph, use the Floyd-Roy-Warshall
algorithm to compute the shortest distance 
$\partial(v_i,v_j)$ between any two pairs of vertices,
$v_i,v_j\in V$. This data determines the entries of
the distance matrices.

%
\begin{center}
\fontsize{9pt}{9pt}
\selectfont
\tt
\begin{lstlisting}

def distance_matrices(Gamma):
    """
    Returns the distance matrices of a graph.

    INPUT:
        Gamma is a graph.
    OUTPUT:
        The sequence of distance matrices

    EXAMPLES:
        sage: A = matrix([[0,1,1,0,0],[1,0,1,0,0],[1,1,0,1,0],[0,0,1,0,1],[0,0,0,1,0]])
        sage: G = Graph(A, format = "adjacency_matrix", weighted = True)
        sage: print distance_matrices(G)
        [[1 0 0 0 0]
         [0 1 0 0 0]
         [0 0 1 0 0]
         [0 0 0 1 0]
         [0 0 0 0 1], 
         [0 1 1 0 0]
         [1 0 1 0 0]
         [1 1 0 1 0]
         [0 0 1 0 1]
         [0 0 0 1 0],
         [0 0 0 1 0]
         [0 0 0 1 0]
         [0 0 0 0 1]
         [1 1 0 0 0]
         [0 0 1 0 0],
         [0 0 0 0 1]
         [0 0 0 0 1]
         [0 0 0 0 0]
         [0 0 0 0 0]
         [1 1 0 0 0]]
    """
    g = Gamma.diameter()
    V = Gamma.vertices()
    n = len(V)
    D = G.distance_all_pairs()
    dist_mats = []
    for i in range(g+1):
        dist_mati = [[0 for j in range(n)] for k in range(n)]
        for v1 in V:
            for v2 in V:
                if D[v1][v2]==i:
                    dist_mati[v1][v2] = 1
        dist_mats.append(matrix(dist_mati))
    return dist_mats

\end{lstlisting}
\end{center}
%

The {\it distance matrix} $D(G)$ of $G$ is
the $n\times n$ matrix

\[
D(G)=(\partial(v_j,v_j))_{1\leq i,j\leq n}.
\]
The distance matrix arises in several applications, including
communication network design (see for example
\cite{GrahamPollak1971})
and network flow algorithms (see for example \cite{Dijkstra1959}).

To compute the distance matrix of a graph, is very 
similar to computing the distance matrices.
Again, use the Floyd-Roy-Warshall
algorithm to compute the shortest distance 
$\partial(v_i,v_j)$ between any two pairs of vertices,
$v_i,v_j\in V$. This data determines the entries of
the distance matrix.

% R. L. Graham and H. O. Pollak, {\it on the addressing problem for
% loop switching}, Bell Tech J 50(1971)2495-2519.
% and
% E. W. Dijkstra, {\it A note on two problems in connection with
% graphs}, Numer. Math. 1 (1959)269-271.
% [(this is [Dijkstra1959] )!
\begin{remark}
{\rm
Thanks to R. Graham and H. Pollak, the following unusual fact is
known:
If $T$ is any tree then 

\[
\det D(T)=(-1)^{n-1}(n-1)2^{n-2}, 
\]
where $n$ denotes the number of vertices of $T$.
In particular, the determinant of the distance matrix of a tree is
independent of the structure of the tree.  This fact is proven in the
paper \cite{GrahamPollak1971}, but see also 
\cite{EdelbergEtAl1976}.
}
\end{remark}
% this lemma is proven in the above paper by Graham-Pollak
% see however:
% M. Edelberg, M. R. Garey, R. L. Graham, {\it On the
% distance matrix of a tree}, Discrete Math. 14 (1976)23-39.


%
\begin{center}
\fontsize{9pt}{9pt}
\selectfont
\tt
\begin{lstlisting}

def distance_matrix(Gamma):
    """
    Returns the distance matrix of a graph.

    INPUT:
        Gamma is a graph.
    OUTPUT:
        The distance matrix

    EXAMPLES:
        sage: A = matrix([[0,1,1,0,0],[1,0,1,0,0],[1,1,0,1,0],[0,0,1,0,1],[0,0,0,1,0]])
        sage: G = Graph(A, format = "adjacency_matrix", weighted = True)
        sage: distance_matrix(G)
        [0 1 1 2 3]
        [1 0 1 2 3]
        [1 1 0 1 2]
        [2 2 1 0 1]
        [3 3 2 1 0]
    """
    g = Gamma.diameter()
    V = Gamma.vertices()
    n = len(V)
    D = G.distance_all_pairs()
    for i in range(g+1):
        dist_mat = [[0 for j in range(n)] for k in range(n)]
        for v1 in V:
            for v2 in V:
                dist_mat[v1][v2] = D[v1][v2]
    return matrix(dist_mat)

\end{lstlisting}
\end{center}
%

%%-----------------------------------------------------------------------%%
%%--- Vertex and edge connectivity --------------------------------------%%

\section{Vertex and edge connectivity}

\begin{itemize}
\item vertex-cut and cut-vertex

\item cut-edge or bridge

\item vertex and edge connectivity
\end{itemize}

\begin{theorem}
\textbf{Menger's Theorem.}
Let $u$ and $v$ be distinct, non-adjacent vertices in a graph
$G$. Then the maximum number of internally disjoint $u$-$v$ paths in
$G$ equals the minimum number of vertices needed to separate $u$ and $v$.
\end{theorem}

\begin{theorem}
\textbf{Whitney's Theorem.}
Let $G = (V, E)$ be a connected graph such that $|V| \geq 3$. Then $G$
is $2$-connected if and only if any pair $u,v \in V$ has two
internally disjoint paths between them.
\end{theorem}


%%-----------------------------------------------------------------------%%
%%--- Centrality of a vertex --------------------------------------------%%

\section{Centrality of a vertex}

\begin{itemize}
\item degree centrality

\item betweenness centrality

\item closeness centrality

\item eigenvector centrality
\end{itemize}


%%-----------------------------------------------------------------------%%
%%--- Network reliability -----------------------------------------------%%

\section{Network reliability}

\begin{itemize}
\item Whitney synthesis

\item Tutte's synthesis of $3$-connected graphs

\item Harary graphs

\item constructing an optimal $k$-connected $n$-vertex graph
\end{itemize}
