%%-----------------------------------------------------------------------%%
%%--- Graph Coloring ----------------------------------------------------%%

\chapter{Graph Coloring}
\label{chap:graph_coloring}

\begin{quote}
\includegraphics[scale=1.0]{image/four-color-conjecture-qed} \\
\noindent
--- Spiked Math,
\url{http://spikedmath.com/210.html}
\end{quote}


%%-----------------------------------------------------------------------%%
%%--- Vertex coloring and chromatic numbers -----------------------------%%

\section{Vertex coloring}

Vertex coloring is a widespread center of interest in graph theory, which has many variants. Formally speaking, a coloring of the vertex set of a graph $G$ is any function $f:V(G)\mapsto \{1,\dots,k\}$ giving to each vertex a color among a set of cardinal $k$. Things get much more difficult when we add to it the constraint under which a coloring becomes a {\it proper} coloring : a coloring with $k$ colors of a graph $G$ is said to be {\it proper} if there are no edges between any two vertices colored with the same color. This can be rephrased in many different ways :
\begin{itemize}
\item $\forall i\in \{1,\dots,k\}, G[f^{-1}(i)]$ is a stable set
\item $\forall u,v\in G,u\neq v, f(u)=f(v)\Rightarrow uv\not \in E(G)$
\item A proper coloring of $G$ with $k$ colors is a partition of $V(G)$ into $k$ independent sets
\end{itemize}

{\bf Chromatic numbers} : quite clearly, it is always possible to find a proper coloring of a graph $G$ using one color per vertex. For this reason, the Coloring Problem is an optimisation problem which amounts to finding the least number $k$ of colors such that $G$ can be properly colored with $k$ colors -- is $k$-colorable. This integer, written $\chi(G)$, is called the chromatic number of $G$.

{\bf Greedy coloring, and an easy upper bound} : the simple fact that Graph Coloring is a NP-complete problem must not prevent one from trying to color it greedily. One such method would be to iteratively pick, in a graph $G$, an uncolored vertex $v$, and to color it with the smallest color available which is not yet used by one of its neighbors (in order to keep it proper). Such a coloring will obviously stay proper until the whole vertex set is colored, and never use more than $\Delta(G)+1$ different colors (where $\Delta(G)$ is the maximal degree of $G$), as in the procedure no vertex will ever exclude more than $\Delta(G)$ colors.

Such an algorithm can be written in Sage in a few lines :

\begin{lstlisting}
sage: g = graphs.RandomGNP(100,5/100)
sage: C = Set(xrange(100))
sage: color = {}
sage: for u in g:
...      interdits = Set([color[v] for v in g.neighbors(u) if color.has_key(v)])
...      color[u] = min(C-interdits)
\end{lstlisting}

\begin{itemize}
\item Brook's Theorem
\item heuristics for vertex coloring
\end{itemize}


%%-----------------------------------------------------------------------%%
%%--- Edge coloring -----------------------------------------------------%%

\section{Edge coloring}

Edge coloring is the direct application of vertex coloring to the line graph of a graph $G$, written $L(G)$, which is the graph whose vertices are the edges of $G$, two vertices being adjacent if and only if their corresponding edges share an endpoint. We write $\chi(L(G)) = \chi'(G)$ the chromatic {\it index} of $G$. In this special case, however, the optimization problem defined above, though still NP-Complete, is much better understood through Vizing's theorem.

\begin{theorem}[Vizing]
The edges of a graph $G$ can be properly colored using at least $\Delta(G)$ colors and at most $\Delta(G)+1$
\end{theorem}
Notice that the lower bound can be easily proved : if a vertex $v$ has a degree $d(v)$, then at least $d(v)$ colors are requied to color $G$ as all the edges incident to $v$ must receive different colors. Besides, the upper bound of $\Delta(G)+1$ can not be deduced from the greedy algorithm given in the previous section, as the maximal degree of $L(G)$ is not equal to $\Delta(G)$ but to $\displaystyle \max_{u\sim v}d(u)+d(v)-2$, which can reach $2\Delta(G)-2$ in regular graphs.


\begin{itemize}
\item algorithm for edge coloring by maximum matching
\item algorithm for sequential edge coloring
\end{itemize}


%%-----------------------------------------------------------------------%%
%%--- Applications of graph coloring ------------------------------------%%

\section{Applications of graph coloring}

\begin{itemize}
\item assignment problems

\item scheduling problems

\item matching problems

\item map coloring and the Four Color Problem
\end{itemize}
